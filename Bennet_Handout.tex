\documentclass[12pt,a4paper,twocolumn]{article}
\usepackage[utf8]{inputenc}
\usepackage[ngerman]{babel}
\usepackage[T1]{fontenc}
\usepackage{amsmath}
\usepackage{amsfonts}
\usepackage{amssymb}
\usepackage{geometry}
\geometry{a4paper,left=15mm,right=15mm, top=1cm, bottom=2cm} 
\author{Tim Gabriel, Bennet Grützner, Philip Hildebrandt, Jochen Jacobs, Loenard König}
\title{Handout Normalverteilung}
\begin{document}

\maketitle

\section{Standardisierung (auch Z-Transformation)}
Bei der Standardisierung wird eine Zufallsvariable z.B. $X$ so transformiert werden, dass die resultierende Zufallsvariable als Erwartungswert $E(X) = 0$ und als Varianz $V(X)=1$ hat. Dies wird bewirkt, indem folgende Schritte ausgeführt werden:
\begin{enumerate}
\item Verschiebung des Erwartungswertes auf der Y-Achse:
\begin{eqnarray}
  Z = X - \mu
\end{eqnarray}
\item Stauchung jeder Zufallsvariable auf die selbe Varianz:
\begin{eqnarray}
  Z = \frac{X - \mu}{\sigma}
\end{eqnarray}
\end{enumerate}
Zur Erhaltung der Fläche gilt zusätzlich noch:
\begin{eqnarray}
	Y = \sigma \cdot B_{n;p}(Z)
\end{eqnarray}

%\textbf{HIER FEHLT NOCH DER BEWEIS VON ERWARTUNGSWERT UND VARIANZ}


\section{Lokaler Grenzwertsatz}

Nach \textrm{\textsc{Gau}ß} ist eine dazu passende Funktion $\varphi$:
\begin{eqnarray}
\varphi(x)=b\cdot e^{-a\cdot\ x^2} \label{eq_gauss}
\end{eqnarray}
(Beweis entfällt \-- Ausschlussprinzip)
Wir benötigen also nur noch passende Werte f"ur $a$ und $b$. Zunächst $a$:\\ 

Durch unsere Vereinheitlichung der Breite ist das $1\cdot \sigma$-Intervall $[-1;1]$. Betrachten des Graphen der Binomialverteilung zeigt uns, an $x=\pm 1$ sind Wendestellen.\\
Für $x=\pm 1$:\\

2. Ableitung: $2abe^{-ax^2} (2ax^2-1)$\\
3. Ableitung: $-4 a^{2} b x e^{-a x^2}(2 a x^2 -3)$\\

Notwendige Bedingung für Wendestelle bei $\pm 1$: $\varphi''(\pm 1)=0$\\
\begin{eqnarray}
2abe^{-a} (2a-1) = 0 \\
\Rightarrow a = 0 \lor a = \frac{1}{2}
\end{eqnarray}

Hinreichende Bedingung für Wendestelle bei $\pm 1$: $\varphi'''(\pm 1)\neq 0$
\begin{eqnarray}
\mp 4 a^{2} b e^{-a}(2 a -3) \neq 0
\end{eqnarray}
Für $a=\frac{1}{2}$ ist dies erfüllt, für $a=0$ kann keine Aussage getroffen werden (muss man aber auch nicht).

Für die Berechnung von $b$ ziehen wir in Betracht, dass der Flächeninhalt unter dem Graphen $1$ sein muss:
\begin{eqnarray}
1&=&\int\displaylimits^{\infty}_{-\infty} b \cdot e^{-\frac{1}{2}x^2} {\mathrm d}x \\
\Leftrightarrow \frac{1}{b}&=&\int\displaylimits^{\infty}_{-\infty} e^{-\frac{1}{2}x^2} {\mathrm d}x \\
I^2 &=& \int\displaylimits^{\infty}_{-\infty} \int\displaylimits^{\infty}_{-\infty} e^{-\frac{1}{2}x^2} e^{-\frac{1}{2}y^2} {\mathrm d}x \, {\mathrm d}y \\
&=& \int\displaylimits^{\infty}_{-\infty} \int\displaylimits^{\infty}_{-\infty} e^{-\frac{1}{2}(x^2 + y^2)} {\mathrm d}x \, {\mathrm d}y \\ 
&=& \int\displaylimits^{2 \pi}_{0} \int\displaylimits^{\infty}_{0} e^{-\frac{1}{2}r^2} d\varphi \, {\mathrm d}r \\
&=& 2 \pi \int\displaylimits^{\infty}_{0} e^{-\frac{1}{2}r^2} {\mathrm d}r \\
&=& 2 \pi \begin{bmatrix}-e^{-\frac{1}{2} r^2} \end{bmatrix}^\infty_0 = 2 \pi\\
\Rightarrow b&:=&\frac{1}{\sqrt{2\pi}}
\end{eqnarray}
Die Formel der Dichtefunktion $\varphi$ lautet also:
\begin{eqnarray}
 \varphi(x) = \frac{1}{2\pi} \cdot e^{-\frac{1}{2}x^2}
\end{eqnarray}




 \section{Integralwertsatz von DeMoivre und Laplace}

\begin{eqnarray}
\lim\limits_{n \rightarrow \infty} P \left(\frac{X_n - n\cdotp}{\sqrt{n\cdotp\cdot(1 - p)}}\right) = \phi(x)
\end{eqnarray}

Allgemein:
\begin{eqnarray}
\lim\limits_{n \rightarrow \infty} P \left(\frac{X_n - \mu}{\sigma}\right) = \phi(x)
\end{eqnarray}

Dieses $\phi$ ist eine Integralfunktion zu $\varphi$. Das folgt mithilfe der Standardisierung aus der Definition der $\varphi$-Funktion. 

\subsection{Kurze Plausibilitätserklärung:}

Die $\varphi$-Funktion stellt den Grenzwert der Binominalverteilung in Unendliche dar. Das heißt aber, der konkrete Wert wird 0. Aus Aufsummieren von unendlich Nullen haben wir beim Integral gehabt. Da war allerdings die Breite eines Wertes Null, nicht die \glqq Höhe\grqq. Diese Umwandlung liefert uns hier die Dichtefunktion. Deswegen nehmen wir das Integral der Dichtefunktion.

\section{Nutzung der Integralfunktion}
Die Nutzung folgt aus dem Integralwertsatz von DeMoivre und Laplace. Sie erfolgt mithilfe von Tabellen. Es ist unmöglich, eine konkrete Stammfunktion anzugeben (einen geschlossenen mathematischen Term). Wer will, kann es ja mal versuchen.
\subsection{Nutzung der Tabelle:}

Die erste Nachkommastelle so wie alles vor dem Komma steht in der horizontalen, die zweite Nachkommastelle in der vertikalen. 
\\
Der nötige Parameter kann wie folgt ausgerechnet werden:\\
{\center $X_n$ ist gegeben,\\
$n$ ist gegeben,\\
$p$ ist gegeben\\}
$$
\mu = n \cdot p
$$
$$
\sigma = n \cdot p \cdot (1 - p)$$


Der Parameter ist dann:

$$
x = \frac{X_n - \mu}{\sigma}
$$

Und schlussendlich:
$$
P(X \le X_n) = \phi(x)
$$


\section{$k\cdot\sigma$-Intervalle}
Die Wahrscheinlichkeit $P$, dass ein Ereignis in $k \cdot \sigma$-Umgebung von $\mu$ liegt:\\

\begin{eqnarray}
& &P(|X-p|\le k\cdot\sigma)\\
 &=& P(\mu-k\cdot\sigma\le X \le \mu + k\cdot\sigma)\\
&=& P\left(\frac{\mu-k\cdot\sigma - \mu}{\sigma}\le \frac{X - \mu}{\sigma}\le \frac{\mu + k\cdot\sigma - \mu}{\sigma}\right) \nonumber \\
&=& P(-k \le Z \le k)\\
&\approx& \phi(k) - \phi(-k)\\
&=& \phi(k) - (1 - \phi(k))\\
&=& \phi(k) + \phi(k) - 1\\
&=& 2\phi(k)- 1
\end{eqnarray}

Einsetzungen von $ k = {0;1;2;...}$:
\begin{eqnarray}
P(|X-p|\le 1\cdot\sigma) &\approx& 68,27 \%\\
P(|X-p|\le 2\cdot\sigma) &\approx& 95,45 \%\\
P(|X-p|\le 3\cdot\sigma) &\approx& 97,73 \%\\
\text{Vgl. Tschebyschow:} \nonumber \\
P(|X-p|\le 2\cdot\sigma) &\approx& 75 \%\\
P(|X-p|\le 3\cdot\sigma) &\approx& 88 \%
\end{eqnarray}

\section{Ausblick}
%\textbf{FEHLT}
Seien $X_1, X_2, X_3, ...$ eine Folge von Zufallsvariablen, die auf dem selben Wahrscheinlichkeitsraum alle dieselbe Verteilung aufweisen und unabhängig sind. Weiter seien $\mu$ und $\sigma > 0$ existent.

Weiter betrachten wir eine standardisierte Zufallsvariable:
$$ Z_n = \sum_{i = 0}^n(X_i)$$

Dann besagt der Zentrale Grenzwertsatz, dass die Verteilungsfunktion von $Z_n$ für $n \rightarrow \infty$ punktweise gegen die Verteilungsfunktion der Normalverteilung $\phi_{\mu;\sigma}(X)$ konvergiert. 


\onecolumn
\pagebreak

\section*{Anhang: Beispieltabelle von Wikipedia:}
\begin{tabular}{l|l|l|l|l|l|l|l|l|l|l}
z & 0 & 0,01 & 0,02 & 0,03 & 0,04 & 0,05 & 0,06 & 0,07 & 0,08 & 0,09\\ \hline
0,0 & 0,50000 & 0,50399 & 0,50798 & 0,51197 & 0,51595 & 0,51994 & 0,52392 & 0,52790 & 0,53188 & 0,53586\\ \hline
0,1 & 0,53983 & 0,54380 & 0,54776 & 0,55172 & 0,55567 & 0,55962 & 0,56356 & 0,56749 & 0,57142 & 0,57535\\ \hline
0,2 & 0,57926 & 0,58317 & 0,58706 & 0,59095 & 0,59483 & 0,59871 & 0,60257 & 0,60642 & 0,61026 & 0,61409\\ \hline
0,3 & 0,61791 & 0,62172 & 0,62552 & 0,62930 & 0,63307 & 0,63683 & 0,64058 & 0,64431 & 0,64803 & 0,65173\\ \hline
0,4 & 0,65542 & 0,65910 & 0,66276 & 0,66640 & 0,67003 & 0,67364 & 0,67724 & 0,68082 & 0,68439 & 0,68793\\ \hline
0,5 & 0,69146 & 0,69497 & 0,69847 & 0,70194 & 0,70540 & 0,70884 & 0,71226 & 0,71566 & 0,71904 & 0,72240\\ \hline
0,6 & 0,72575 & 0,72907 & 0,73237 & 0,73565 & 0,73891 & 0,74215 & 0,74537 & 0,74857 & 0,75175 & 0,75490\\ \hline
0,7 & 0,75804 & 0,76115 & 0,76424 & 0,76730 & 0,77035 & 0,77337 & 0,77637 & 0,77935 & 0,78230 & 0,78524\\ \hline
0,8 & 0,78814 & 0,79103 & 0,79389 & 0,79673 & 0,79955 & 0,80234 & 0,80511 & 0,80785 & 0,81057 & 0,81327\\ \hline
0,9 & 0,81594 & 0,81859 & 0,82121 & 0,82381 & 0,82639 & 0,82894 & 0,83147 & 0,83398 & 0,83646 & 0,83891\\ \hline
1,0 & 0,84134 & 0,84375 & 0,84614 & 0,84849 & 0,85083 & 0,85314 & 0,85543 & 0,85769 & 0,85993 & 0,86214\\ \hline
1,1 & 0,86433 & 0,86650 & 0,86864 & 0,87076 & 0,87286 & 0,87493 & 0,87698 & 0,87900 & 0,88100 & 0,88298\\ \hline
1,2 & 0,88493 & 0,88686 & 0,88877 & 0,89065 & 0,89251 & 0,89435 & 0,89617 & 0,89796 & 0,89973 & 0,90147\\ \hline
1,3 & 0,90320 & 0,90490 & 0,90658 & 0,90824 & 0,90988 & 0,91149 & 0,91309 & 0,91466 & 0,91621 & 0,91774\\ \hline
1,4 & 0,91924 & 0,92073 & 0,92220 & 0,92364 & 0,92507 & 0,92647 & 0,92785 & 0,92922 & 0,93056 & 0,93189\\ \hline
1,5 & 0,93319 & 0,93448 & 0,93574 & 0,93699 & 0,93822 & 0,93943 & 0,94062 & 0,94179 & 0,94295 & 0,94408\\ \hline
1,6 & 0,94520 & 0,94630 & 0,94738 & 0,94845 & 0,94950 & 0,95053 & 0,95154 & 0,95254 & 0,95352 & 0,95449\\ \hline
1,7 & 0,95543 & 0,95637 & 0,95728 & 0,95818 & 0,95907 & 0,95994 & 0,96080 & 0,96164 & 0,96246 & 0,96327\\ \hline
1,8 & 0,96407 & 0,96485 & 0,96562 & 0,96638 & 0,96712 & 0,96784 & 0,96856 & 0,96926 & 0,96995 & 0,97062\\ \hline
1,9 & 0,97128 & 0,97193 & 0,97257 & 0,97320 & 0,97381 & 0,97441 & 0,97500 & 0,97558 & 0,97615 & 0,97670\\ \hline
2,0 & 0,97725 & 0,97778 & 0,97831 & 0,97882 & 0,97932 & 0,97982 & 0,98030 & 0,98077 & 0,98124 & 0,98169\\ \hline
2,1 & 0,98214 & 0,98257 & 0,98300 & 0,98341 & 0,98382 & 0,98422 & 0,98461 & 0,98500 & 0,98537 & 0,98574\\ \hline
2,2 & 0,98610 & 0,98645 & 0,98679 & 0,98713 & 0,98745 & 0,98778 & 0,98809 & 0,98840 & 0,98870 & 0,98899\\ \hline
2,3 & 0,98928 & 0,98956 & 0,98983 & 0,99010 & 0,99036 & 0,99061 & 0,99086 & 0,99111 & 0,99134 & 0,99158\\ \hline
2,4 & 0,99180 & 0,99202 & 0,99224 & 0,99245 & 0,99266 & 0,99286 & 0,99305 & 0,99324 & 0,99343 & 0,99361\\ \hline
2,5 & 0,99379 & 0,99396 & 0,99413 & 0,99430 & 0,99446 & 0,99461 & 0,99477 & 0,99492 & 0,99506 & 0,99520\\ \hline
2,6 & 0,99534 & 0,99547 & 0,99560 & 0,99573 & 0,99585 & 0,99598 & 0,99609 & 0,99621 & 0,99632 & 0,99643\\ \hline
2,7 & 0,99653 & 0,99664 & 0,99674 & 0,99683 & 0,99693 & 0,99702 & 0,99711 & 0,99720 & 0,99728 & 0,99736\\ \hline
2,8 & 0,99744 & 0,99752 & 0,99760 & 0,99767 & 0,99774 & 0,99781 & 0,99788 & 0,99795 & 0,99801 & 0,99807\\ \hline
2,9 & 0,99813 & 0,99819 & 0,99825 & 0,99831 & 0,99836 & 0,99841 & 0,99846 & 0,99851 & 0,99856 & 0,99861\\ \hline
3,0 & 0,99865 & 0,99869 & 0,99874 & 0,99878 & 0,99882 & 0,99886 & 0,99889 & 0,99893 & 0,99896 & 0,99900\\ \hline
3,1 & 0,99903 & 0,99906 & 0,99910 & 0,99913 & 0,99916 & 0,99918 & 0,99921 & 0,99924 & 0,99926 & 0,99929\\ \hline
3,2 & 0,99931 & 0,99934 & 0,99936 & 0,99938 & 0,99940 & 0,99942 & 0,99944 & 0,99946 & 0,99948 & 0,99950\\ \hline
3,3 & 0,99952 & 0,99953 & 0,99955 & 0,99957 & 0,99958 & 0,99960 & 0,99961 & 0,99962 & 0,99964 & 0,99965\\ \hline
3,4 & 0,99966 & 0,99968 & 0,99969 & 0,99970 & 0,99971 & 0,99972 & 0,99973 & 0,99974 & 0,99975 & 0,99976\\ \hline
3,5 & 0,99977 & 0,99978 & 0,99978 & 0,99979 & 0,99980 & 0,99981 & 0,99981 & 0,99982 & 0,99983 & 0,99983\\ \hline
3,6 & 0,99984 & 0,99985 & 0,99985 & 0,99986 & 0,99986 & 0,99987 & 0,99987 & 0,99988 & 0,99988 & 0,99989\\ \hline
3,7 & 0,99989 & 0,99990 & 0,99990 & 0,99990 & 0,99991 & 0,99991 & 0,99992 & 0,99992 & 0,99992 & 0,99992\\ \hline
3,8 & 0,99993 & 0,99993 & 0,99993 & 0,99994 & 0,99994 & 0,99994 & 0,99994 & 0,99995 & 0,99995 & 0,99995\\ \hline
3,9 & 0,99995 & 0,99995 & 0,99996 & 0,99996 & 0,99996 & 0,99996 & 0,99996 & 0,99996 & 0,99997 & 0,99997\\ \hline
4,0 & 0,99997 & 0,99997 & 0,99997 & 0,99997 & 0,99997 & 0,99997 & 0,99998 & 0,99998 & 0,99998 & 0,99998

\end{tabular}

\end{document}
