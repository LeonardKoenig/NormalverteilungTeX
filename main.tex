\documentclass[14pt]{beamer}
\usetheme{Berkeley}
\usepackage[utf8]{inputenc}
\usepackage[ngerman]{babel}
\usepackage[T1]{fontenc}
\usepackage{amsmath}
\usepackage{amsfonts}
\usepackage{amssymb}
\usepackage{graphicx}
%\usepackage{pdfplots}
%\usepackage{pdfplotstable}
\author{AUTHOR}
\title{TITLE}
%\setbeamercovered{transparent} 
%\setbeamertemplate{navigation symbols}{} 
%\logo{} 
\institute{Herder Gymnasium Berlin} 
\date{} 
%\subject{test} 
\begin{document}

\begin{frame}
\titlepage
\end{frame}

\begin{frame}
\tableofcontents
\end{frame}

\section{Was ist eine Dichtefunktion?}
\begin{frame}{Wozu eine Dichtefunktion?}
\textbf{Wahrscheinlichkeitsfunktion}:

{\small $$ P(X \in\mathbb{A}) \, = \, \sum_{x\in A} \rho(x) $$}
\begin{itemize}

\item eignet sich nur für diskrete Wahrscheinlichkeitsverteilungen:

\begin{itemize}
\only<1>{
\item Binomial
\item Hypergeometrisch
\item Poisson
}
\only<2>{
\item[$\Rightarrow$] keine Aussage möglich für stetige Verteilungen, da $P(X \, = \, x) = 0 \, \forall X \in \mathbb{A}$
}
\end{itemize}
\end{itemize}

\end{frame}

\begin{frame}{Definition}
$ X;a;b \in \mathbb{R}, a<b$\\
$f \colon \mathbb{R} \rightarrow [0,\infty)$ ist die \textbf{Wahrscheinlichkeitsdichte} der Verteilung von $X$, wenn gilt:
$$
P(a\leq X\leq b)=\int_a^bf(x)\,\mathrm dx
$$
\end{frame}



\section{Was sagt der Grenzwertsatz aus?} % (Bedeutung)
\subsection{Beweis} %  od. Plausibilitätsprüfung
\section{Gauß} %Hinführung auf Gauß'sche Formel (e^(-1/2x^2)), so wie es Herr Steinkrauß erläutert hat (oder auch anders, aber in diesem Detailgrad)
\section{Aufgaben zur lokalen Näherung}
\end{document}



Was ich dann (nachfolgend) mache:
- Was sagt das Integral aus?
- Wie bilde ich überhaupt das Integral?
- Wie gucke ich die Werte nach? (Tabellen)
+ ein paar Aufgaben 