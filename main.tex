\documentclass[12pt,a4paper]{article}
\usepackage[utf8]{inputenc}
\usepackage[ngerman]{babel}
\usepackage{amsmath}
\usepackage{amsfonts}
\usepackage{amssymb}
\usepackage[left=2cm,right=4cm,top=3cm,bottom=3cm]{geometry}
\author{Leonard König}
\begin{document}
-
\end{document}
- Was ist die Dichtefunktion?
- Was sagt der Grenzwertsatz? (Bedeutung)
- Beweis od. Plausibilitätsprüfung
- Hinführung auf Gauß'sche Formel (e^(-1/2x^2)), so wie es Herr Steinkrauß erläutert hat (oder auch anders, aber in diesem Detailgrad)
Was ich dann (nachfolgend) mache:
- Was sagt das Integral aus?
- Wie bilde ich überhaupt das Integral?
- Wie gucke ich die Werte nach? (Tabellen)
+ ein paar Aufgaben (lokalen Näherung)